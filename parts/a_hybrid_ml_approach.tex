\begin{frame}{A hybrid ML approach}
\begin{center}
Prototracking $\rightarrow$ Kernel generation $\rightarrow$ CNN to find PVs $\rightarrow$ Informed tracking
\end{center}

\begin{columns}[b]
    \column{.28\textwidth}
    \begin{block}{Prototracking}
    \begin{itemize}
        \item Ultra-simple/fast
        \item Triplets only
        \item Used for kernel only
        \end{itemize}
    \end{block}
    \column{.30\textwidth}
    \begin{block}{Vertexing}
    \begin{itemize}
        \item High efficiency
        \item Low false positive rate
        \item Useful for other reasons
        \end{itemize}
    \end{block}
    \column{.28\textwidth}
    \begin{block}{Tracking}
    \begin{itemize}
        \item Faster (effect TBD)
        \item Uses search windows
        \item Higher efficiency
    \end{itemize}
    \end{block}
\end{columns}
    \begin{block}{Machine learning features (so far)}
        \begin{itemize}
            \item Prototracking converts sparse 3D dataset to feature-rich 1D dataset
            \item Easy and effective visualization due to 1D nature
            \item Can see results with simple unoptimized 2-layer CNN + 1-layer linear
        \end{itemize}
    \end{block}

\vspace{.3em}
\begin{center}
What follows is a proof of principle implementation for finding PVs.
\end{center}
\end{frame}
