% Part 9
\subsection{Future plans}
\begin{frame}{Future plans}
\vskip -0.10in
  \begin{block}{Some ideas}
    \begin{itemize}
        \item 
          Model PV resolution as a function of the number of tracks; right now, the target
          function is always generated assuming $ \sigma_z = 100 \, \mu $m;
        \item
          Extract $ \sigma_z $ from predicted signals;
        \item
          Extend algorithm to find PV ($ x, \, y, \, z $) target functions,
          not just PV $ z $ target functions.
        \item 
        \textcolor{brickred}{Mask PVs with $ < $ 5 long tracks (not labeled as PVs  now)} 
        \item
          Ask the algorithm (very nicely) to find Secondary Vertices as well;
          it should probably use both the original KDE histogram and the 
          learned PV histogram as inputs.
          It may be possible to re-use some of the features generated by the 
          convolutional layers.
        \item
          Integrate  KDE plus PV-finding code into an iterative tracking and vertexing
          algorithm; well-defined vertex positions may be able to serve as anchors for 
          good tracks, restricting the roads to be searched.  
        \item
          Optimize NN architecture to (i) improve learning, (ii) improve learning speed,
          (iii)~minimize inference costs (cycles and memory).
          Increase training sample.
   \end{itemize}
  \end{block}
\end{frame}





