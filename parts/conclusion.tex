% Part 9
\subsection{Future plans}
\begin{frame}{Future plans}
    \begin{itemize}
        \item
          Model PV resolution as a function of the number of tracks; right now, the target
          function is always generated assuming $ \sigma_z = 100 \, \mu $m;
        \item
          Extract $ \sigma_z $ from predicted signals;
        \item
          Extend algorithm to find PV ($ x, \, y, \, z $) target functions,
          not just PV $ z $ target functions.
        \item
          Ask the algorithm (very nicely) to find Secondary Vertices as well;
          it should probably use both the original KDE histogram and the
          learned PV histogram as inputs.
          It may be possible to re-use some of the features generated by the
          convolutional layers.
        \item
          Integrate  KDE plus PV-finding code into an iterative tracking and vertexing
          algorithm; well-defined vertex positions may be able to serve as anchors for
          good tracks, restricting the roads to be searched.
        \item
          Optimize NN architecture to (i)~improve learning, (ii)~improve learning speed,
          (iii)~minimize inference costs (cycles and memory).
          Increase training sample.
    \end{itemize}
\end{frame}

\subsection{Questions}
\begin{frame}{Questions for ATLAS and CMS}
    \begin{itemize}
      \item
          Beam width ($x$, $y$): $\unit[40]{\mu m}$ for LHCb; what are the widths for ATLAS and CMS?
      \item
          Transverse resolution: $\unit[5\textup{--}15]{\mu m}$ for LHCb depending on number of tracks; what is it for ATLAS and CMS?
      \item
          Longitudinal resolution: $\unit[40\textup{--}100]{\mu m}$ for LHCb depending on number of tracks; what is it for ATLAS and CMS?
      \item
          Cleaning up prototracks based on IP could simplify kernel
      \item
          Can prototracking be done in the ATLAS and CMS triggers?
    \end{itemize}
\end{frame}
